\documentclass{article}

\usepackage{mypckg}

\title{Assignment 4}
\author{Exercise 2}
\date{}

\begin{document}

\maketitle

Introductory courses on mathematical reasoning usually include the topics of truth tables and proofs by contradiction. We display one such truth table below, defining the logic operation of the Sheffer stroke.
\begin{center}
\truthtable{A \uparrow B}{T}{T}{T}{F}
\end{center}
The Sheffer stroke alone can be used to create each of the logic operations of and, or, not, and implies. For example, we display two equivalent truth tables representing "or", centered and separated by a \verb#\qquad#:
\begin{center}
\truthtable{A \lor B}{F}{T}{T}{T} \qquad \truthtable{\left(A \uparrow B\right) \uparrow \left(B \uparrow B\right)}{F}{T}{T}{T}
\end{center}

We display one of the oldest known proofs by contradiction in the proof of \hyperlink{Theorem 1}{1}.

\theorem{\hypertarget{Theorem 1}{There is no largest prime.}} \\

\begin{proofbyc}
    If $p$ is the largest prime, then all primes divide $p!$. This implies $p!+1$ does not have a prime divisor. \hfill
\end{proofbyc}

\begin{huge}
    \begin{center}
        TIC-TAC-TOE!

        \tictactoe{X}{O}{X}{O}{X}{}{O}{O}{X}
    \end{center}
\end{huge}
\end{document}
