\documentclass[11pt]{article}

\usepackage{351package}

\title{Customization and tables}
\author{Math 351}
\date{}

\begin{document}

\maketitle

Almost everything in \LaTeX{} can be customized, including \LaTeX{} commands
themselves.  The standard fonts, margin sizes, and so on, were carefully designed to 
make technical documents readable, so changing the out-of-the-box behavior is not 
always wise.  Nonetheless, there are times when it may be appropriate to customize.

\section{Font size and face}

To change to a 11 point font, use the \verb~[11pt]~ option in the initial 
\verb~\documentclass~ command in this way:
\begin{verbatim}
\documentclass[11pt]{article}
\end{verbatim}
This document is typed in 11pt font.  The font size can also be 10pt (the default 
option) or 12pt.  Another option in the \verb~\documentclass~ command is 
\verb~twocolumn~, which has the effect of producing two columns per page.  

The font face can be changed by including certain packages in the preamble of the 
document.  For example, to change both the text and mathematics to the ``Times'' font, 
include the command \verb~\usepackage{mathptmx}~ before the \verb~\begin{document}~.
There are more cutting edge ways to change the font using XeLaTeX (which we will see later in the 
course), but for now these packages are still used by many. 


Most versions of \LaTeX{} have a number of pre-installed fonts, some of which can be 
selected by calling these packages:
\begin{center}
\begin{tabular}{|lllll|}
\hline
\textbf{Package} & \textbf{Roman} & \textbf{Math} & \textbf{Sans serif} & \textbf{Typewriter} \\
\hline
\hline	
(none) &  CM Roman &  CM Roman & CM Sans  & CM Typewriter \\	
mathpazo &  Palatino &  Palatino &  &  \\
mathptmx &  Times &  Times &  &  \\
helvet &  &  &  Helvetica&  \\
avant &  &  &  Avant Garde &  \\
courier  &  &  &  &  Courier  \\
chancery &  Chancery  &  &  &  \\
bookman  &  Bookman  &  &  Avant Garde &  Courier  \\
newcent  &  New Century  &  &  Avant Garde &  Courier  \\
charter  &  Charter  &  &  &  \\
\hline
fourier  &  Utopia&  Fourier  &  &  \\
eulervm  &  &  Euler &  &  \\ 
\hline
\end{tabular}
\end{center}

An empty entry indicates that a package does not have an effect on a given font face.  
The last two font selections are listed separately because they are not usually found 
in the basic version of \LaTeX{}, but are given by the ``texlive-full'' version.  
Later in the course we will show how to include arbitrary font faces.

To select the roman font, math font, sans serif font, and typewriter font separately, 
include consecutive \verb~\usepackages~ in a correct order.  For example, the commands
\begin{verbatim}
\usepackage[sc]{mathpazo}
\usepackage{charter}
\usepackage[scaled]{helvet}
\end{verbatim}
uses the Palatino math font, Charter roman font, \textsf{Helvetica sans serif font} 
(invoked by \verb~\textsf{..}~), and Computer Modern Typewriter font (as seen 
in \verb~verbatim~ statements).  These are the choices made for this document.  
The \verb~[sc]~ option on \verb~mathpazo~ gives a better small caps font, and the 
\verb~[scaled]~ option on \verb~helvet~ gives an adjusted version of the Helvetica 
font.

To set the default font face for the entire document to the sans serif font, say, then
place \verb~\renewcommand{\familydefault}{\sfdefault}~ in the preamble.  If using this
command, then roman font can be invoked using \verb~\textrm{..}~. 

\section{Margins and spacing}

The margins of the document can be controlled with the \verb~geometry~ package.  To 
set the left, top, right, and bottom margins to specific values, place a command such 
as 
\begin{verbatim}
\usepackage[left=35mm,top=2cm,right=35mm,bottom=2cm]{geometry}
\end{verbatim}
in the preamble.  The margins used in this document are those values shown above.  As 
another example,
\begin{verbatim}
\usepackage[landscape, margin=2in]{geometry}
\end{verbatim}
changes all margins to be 2 inches and prints in landscape mode.

It usually bad form to manually adjust the vertical or horizontal spacing inside the 
body of the document when writing an article or book, but it might be appropriate to 
manually force white space when creating documents such as syllabi, exams, or resumes. 

The line break command \verb~\\~ has an option to add extra space; to add an 
extra \verb~1cm~, use \verb~\\[1cm]~.  Alternatively, to force an extra vertical space 
of \verb~1cm~ between two paragraphs, the \verb~\vspace{1cm}~ command can be used.  
Sometimes \LaTeX{} will think adding a vertical space using \verb~\vspace~ is a bad 
idea and won't cooperate; such a vertical space can be demanded with the 
command \verb~\vspace*{1cm}~.

The \verb~\vfill~ command produces a length which can stretch or shrink vertically, 
pushing the text after the \verb~\vfill~ command as far down the page as possible.  
This command can be used in tandem with the \verb~\newpage~ command, which forces a 
new page to begin.  For instance, if writing a mathematics exam, the \LaTeX{} commands
\begin{verbatim}
\begin{problem} Evaluate \( \int \ln x \, dx \). \end{problem}
\vfill
\begin{problem} Evaluate \( \int \sin x \, dx \). \end{problem}
\vfill
\newpage
\end{verbatim}
will produce the next page in the document (provided the \verb~amsthm~ package is 
loaded and \verb~\theoremstyle{definition}~ and \verb~\newtheorem{problem}{}~ both 
appear in the preamble).

\newpage

\begin{problem} Evaluate \( \int \ln x \, dx \). \end{problem}
\vfill
\begin{problem} Evaluate \( \int \sin x \, dx \). \end{problem}
\vfill

\newpage

Analogous to the \verb~\vspace~, \verb~\vspace*~, and \verb~\vfill~ commands are the 
\verb~\hspace~, \verb~\hspace*~, and \verb~\hfill~ commands, which produce horizontal 
space, a forced horizontal space, and a rubber horizontal fill.  

\section{Newcommands and style files}

\LaTeX{} commands themselves can be customized.  To create a new command, place
\begin{verbatim}
\newcommand{\name}{definition}
\end{verbatim}
in the preamble.  Then, to call the command, use \verb~\name~ in the document.  The 
compiler will complain if \verb~\name~ is a predefined \LaTeX{} command.  
The \verb~\newcommand~ command has an optional parameter to include input.  For 
instance,
\begin{verbatim}
\newcommand{\integral}[2]{\( \int_{\mathbb{R}} #1 \, d#2 \)}
\end{verbatim}
defines the command \verb~\integral~ which takes in 2 inputs, placed where the 
\verb~#1~ and \verb~#2~ appear in the definition.  If this command was in the preamble,
then calling the command \verb~\integral{\sin x}{x}~ would 
produce \integral{\sin x}{x}.  There can be up to 9 inputs.

For function names in math mode which are not predefined (such 
as $\ln x$, $\sin x$, $\arctan x$), use the command such as  
\verb~\DeclareMathOperator{\dimension}{dim}~ 
to define a command \verb~\dimension~.  This produces ``$\dimension$'', a math mode 
symbol.

To redefine a previously defined \LaTeX{} command, use the syntax
\begin{verbatim}
\renewcommand{\old}{\new}
\end{verbatim}
For instance, \verb~\renewcommand{\phi}{\varphi}~ changes the appearance of $\phi$ 
throughout the document.  As another example, this command can be used to change the 
end-of-proof symbol in the \verb~amsthm~ proof environment into creating a black 
square pushed to the right of the line by including
\verb~\renewcommand{\qed}{\hfill \( \blacksquare \)}~
in the preamble.

Analogous to \verb~\newcommand~, there is a \verb~\newenvironment~ command to create
custom environments.  The syntax to be placed in the preamble is 
\begin{verbatim}
\newenvironment{name}{before}{after}
\end{verbatim}
To call the command, use \verb~\begin{name} .. \end{name}~.  Then, commands in 
\verb~before~ are run before \verb~..~ and commands in \verb~after~ are run after
\verb~..~.  Just like \verb~\newcommand~, there is an option for up to $9$ input 
variables.  For example, the command
\begin{verbatim}
\newenvironment{tinytext}{tiny text:\begin{tiny}}{\end{tiny}!}
\end{verbatim}
can be used to create \begin{tinytext}This text really is quite small\end{tinytext}

After loading packages, choosing the font, setting margins, and defining new commands
and new environments, the preamble can become lengthy.  Especially when reusing the 
same preamble for multiple documents, it may be convenient to store the preamble
in a file with the \verb~.sty~ extension (the \verb~.sty~ stands for ``style file'').
In the first line of the style file, place the command 
\begin{verbatim}
\ProvidesPackage{351package}
\end{verbatim}
and then call the package just like any other package with the 
\verb~\usepackage~ command in the preamble of the main document.

\section{Tables}

\LaTeX{} is not the software of choice to make complicated, detailed spreadsheets, 
but for simple and short tables there is the tabular environment.  The syntax
to create a table is 
\begin{verbatim}
\begin{tabular}{column specification}
.. & .. & .. \\
.. & .. & .. \\
\end{tabular}
\end{verbatim}
where \verb~column specification~ is a string containing these characters:
\begin{center}
\begin{tabular}{cl}
\verb~l~ & for a column of left aligned text, \\
\verb~c~ & for a column of centered text, \\
\verb~r~ & for a column of right aligned text, \\ 
\verb~|~ & to create a vertical bar, or \\
\verb~p{width}~ & for a column with wraparound text of length \verb~width~.
\end{tabular}
\end{center}
The syntax for moving to the next column and starting a new line is the same 
syntax as when using the \verb~\begin{matrix}..\end{matrix}~ command in math mode.

To create a horizontal line in the \verb~tabular~ environment, use the command
\verb~\hline~.  We end with one more example of a \verb~tabular~ command, giving the
addition table for the additive group $\mathbb{Z}_4$:
\begin{center}
\begin{tabular}{c|cccc}
$\mathbb{Z}_4$ & $0$ & $1$ & $2$ & $3$ \\
\hline 
$0$ & $0$ & $1$ & $2$ & $3$ \\
$1$ & $1$ & $2$ & $3$ & $0$ \\
$2$ & $2$ & $3$ & $0$ & $1$ \\
$3$ & $3$ & $0$ & $1$ & $2$ \\
\end{tabular}
\end{center}
There are many more examples of tables of varying complexity on pages 45--48 
of the text.  Later, when we discuss packages, we will introduce the wonderful booktabs 
environment for better looking tables. 

\end{document}
