\documentclass{article}
\usepackage[german]{babel}
\begin{document}
As shown on page 5 of the textbook, special characters such as \$, \{, \~{}, and \%
are produced with a preceding backslash. Another way to produce these char-
acters is to use the \verb!\verb#stuff#!{} command, which typesets ``stuff'' verbatim
(including spaces and special characters) in a typewriter font. The beginning
and ending \verb\#\{} delimiters can be replaced with other non-letter characters such as \^{}, \textbackslash, 4, or !.

Only use \verb#\verb# to display short strings verbatim. Do not use \verb#\verb# to
change the font. For that purpose there is the command \verb#\texttt{text}# which
prints ``text'' in a typewriter font.

There are other commands which change the font: \textbf{bold}, \textsf{sans serif}, \textsl{slanted},
\textit{italicized}, and \textsc{small caps} are produced by \verb#\textbf{}#, \verb#\textsf{}#, \verb#\textsl{}#,
\verb#\textit{}#, and \verb#\textsc{}#. Within any of these fonts, words can be \emph{emphasized}
using \verb#\emph{}#. For instance, \textbf{this is \emph{special} bold text}. Text can also be
\underline{underlined} with \verb#\underline{}#.

Use special fonts sparingly, if at all. The user should focus on content and
let the \LaTeX compiler do the typesetting.

The compiler tries to align the first and last characters in consecutive lines
in a paragraph. As a result, the space between words can vary from line to
line. Na"ive \LaTeX{}ers sometimes try to change this spacing by forcing breaks
with commands such as \verb#\newline# or \textbackslash{}\textbackslash{}. Don't do this.

Some users may try to change the spacing between paragraphs using com-
mands such as \textbackslash{}\textbackslash{}\verb#[4cm]# or \verb#\vspace{1.1in}#. These last commands produce
vertical spaces of 4 centimeters and 1.1 inches, resectively. Their use is dis-
couraged.
\end{document}
