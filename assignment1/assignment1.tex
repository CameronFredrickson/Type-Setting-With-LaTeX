\documentclass{article}
\usepackage[german]{babel}
\usepackage[super]{nth}
\usepackage{textcomp}
\begin{document}

\begin{flushleft}
\textbf{Exercise 1:}
\end{flushleft}

As shown on page 5 of the textbook, special characters such as \$, \{, \~{}, and \%
are produced with a preceding backslash. Another way to produce these char-
acters is to use the \verb!\verb#stuff#!{} command, which typesets ``stuff'' verbatim
(including spaces and special characters) in a typewriter font. The beginning
and ending \verb\#\{} delimiters can be replaced with other non-letter characters such as \^{}, \textbackslash, 4, or !.

Only use \verb#\verb#{} to display short strings verbatim. Do not use \verb#\verb#{} to
change the font. For that purpose there is the command \verb#\texttt{text}#{} which
prints ``text'' in a typewriter font.

There are other commands which change the font: \textbf{bold}, \textsf{sans serif}, \textsl{slanted},
\textit{italicized}, and \textsc{small caps} are produced by \verb#\textbf{}#, \verb#\textsf{}#, \verb#\textsl{}#,
\verb#\textit{}#, and \verb#\textsc{}#. Within any of these fonts, words can be \emph{emphasized}
using \verb#\emph{}#. For instance, \textbf{this is \emph{special} bold text}. Text can also be
\underline{underlined} with \verb#\underline{}#.

Use special fonts sparingly, if at all. The user should focus on content and
let the \LaTeX{} compiler do the typesetting.

The compiler tries to align the first and last characters in consecutive lines
in a paragraph. As a result, the space between words can vary from line to
line. Na"ive \LaTeX{}ers sometimes try to change this spacing by forcing breaks
with commands such as \verb#\newline# or \verb#\\#{}. Don't do this.

Some users may try to change the spacing between paragraphs using com-
mands such as \verb#\\[4cm]# or \verb#\vspace{1.1in}#. These last commands produce
vertical spaces of 4 centimeters and 1.1 inches, respectively. Their use is dis-
couraged.

\begin{flushleft}
\textbf{Exercise 2:}
\end{flushleft}

``Hello? hellooo, HELL0-O-0? \ldots'' she screamed at the mike.
``the mike'' was the host of the party and was still in the back of the ballroom
setting up tables and chairs. the mike was a very particular individual described
as a good-looking, quick-thinking, bad-tempered man in his mid to late thirties. He was infamous for giving unsuspecting individuals the $\nth{3}^{\circ}$ and could deliver the iceiest stare with a windchill of -40\textdegree{}F. Paticularly excited to take part in a heated interaction he screamed back ``ENCHANT\'E!''.
\end{document}
