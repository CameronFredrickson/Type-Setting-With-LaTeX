\documentclass{article}

\usepackage{amssymb,amsmath,amsthm}

\title{Assignment 2}
\author{Cameron Fredrickson}
\date{}

\begin{document}

\maketitle

\noindent \textbf{Exercise 1}

\noindent \rule{\textwidth}{.1mm}

Every mathematical statement, such as $ \sum_{n=1}^\infty 1/n^2 = \pi^2/6$, should be part of a sentence. Even equations need punctuation!

The notation $ \lim_{x \to a} f(x) = L$ means that for every $ \varepsilon > 0$ there is a $ \delta > 0$ such that $|x-a| < \delta$ implies $|f(x)-L| < \varepsilon$. An incorrect way to typeset this definition is \begin{equation*}
\forall \left( \varepsilon > 0 \right) \exists \left( \delta > 0 \right) \ni \left( |x-a| < \delta \Longrightarrow |f(x)-L| < \varepsilon \right).
\end{equation*}

The symbols $\forall$, $\exists$, $\ni$, and $\Longrightarrow$ should be used only in the context of the mathematical subject of formal logic and should not replace the words ``for all'', ``exists'', and ``such that'', and ``implies''.

One of your instructor's favorite mathematical statements is \begin{equation*}
n! \sim \sqrt{2 \pi n} \left( \frac{n}{e} \right)^{n},
\end{equation*} otherwise known as Stirling's formula. As an example,   100! is approximately equal to $ \sqrt{200 \pi} \left( 100/e \right)^{100} \approx 9.32 * 10^{157}.$



\noindent \textbf{Exercise 2}

\noindent \rule{\textwidth}{.1mm}
\end{document}