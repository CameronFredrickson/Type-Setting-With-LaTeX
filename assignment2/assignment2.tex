\documentclass{article}

\usepackage{amssymb,amsmath,amsthm}
\usepackage{esint}
\usepackage[latin]{babel}

\title{Assignment 2}
\author{Cameron Fredrickson}
\date{}

\begin{document}

\maketitle

\noindent \textbf{Exercise 1}

\noindent \rule{\textwidth}{.1mm}

Every mathematical statement, such as $ \sum_{n=1}^\infty 1/n^2 = \pi^2/6$, should be part of a sentence. Even equations need punctuation!

The notation $ \lim\limits_{x \rightarrow a} f(x) = L$ means that for every $ \varepsilon > 0$ there is a $ \delta > 0$ such that $|x-a| < \delta$ implies $|f(x)-L| < \varepsilon$. An incorrect way to typeset this definition is \begin{equation*}
\forall \left( \varepsilon > 0 \right) \exists \left( \delta > 0 \right) \ni \left( |x-a| < \delta \Longrightarrow |f(x)-L| < \varepsilon \right).
\end{equation*}

The symbols $\forall$, $\exists$, $\ni$, and $\Longrightarrow$ should be used only in the context of the mathematical subject of formal logic and should not replace the words ``for all'', ``exists'', and ``such that'', and ``implies''.

One of your instructor's favorite mathematical statements is \begin{equation*}
n! \sim \sqrt{2 \pi n} \left( \frac{n}{e} \right)^{n},
\end{equation*} otherwise known as Stirling's formula. As an example,   100! is approximately equal to $ \sqrt{200 \pi} \left( 100/e \right)^{100} \approx 9.32 \times 10^{157}.$

The following is true: \begin{align*}
\left| \int_1^a \frac{\sin x}{x} \, dx\right| &\leq \int_1^a \left|\frac{\sin x}{x}\right| \, dx \\
&\leq \int_1^a \frac{1}{x} \, dx \\
&= \ln a.
\end{align*}

After first simplfying using the exponential and the natural log functions, L'H\^opital's rule can be used to evaluate $ \lim\limits_{x \rightarrow 2^-} (4-x)^{1/(2-x)}$.

Take $\mathbf{x, y} \in \mathbb{R}^n$. The inner product of $\mathbf{x}$ and $\mathbf{y}$ is defined by $\langle \mathbf{x},\mathbf{y} \rangle = \mathbf{x}^\intercal \mathbf{y}$. It follows that \begin{equation*}
\langle \mathbf{x},\mathbf{x} \rangle = \mathbf{x}^\intercal \mathbf{x} = \Vert \mathbf{x} \Vert^2,
\end{equation*} which is a non-negative real number.

\newpage
\noindent \textbf{Exercise 2}

\noindent \rule{\textwidth}{.1mm}

\begin{center}
\textbf{1. Navier Stokes Equations}
\end{center}

\begin{equation*}
\frac{\partial \mathbf{v}}{\partial t} + \left( \mathbf{v} \cdot \nabla \right) \mathbf{v} =  - \nabla p + v \Delta \mathbf{v} + \mathbf{f}\left( \mathbf{x}, t\right)
\end{equation*}

\begin{center}
\textbf{2. Additive Identity or O(x)}
\end{center}

\begin{center}
\textbf{3. Green's Theorem}
\end{center}

A method of surface integration, Green's Theorem allows you to integrate over a region D enclosed by a curve C in a counter-clockwise direction. (What are P and Q?)
\begin{equation*}
\ointctrclockwise\limits_C \left( P \, dx + Q \, dy \right) = \iint\limits_D \left( \frac{\partial P}{\partial x} - \frac{\partial Q}{\partial y} \, dx \, dy \right)
\end{equation*}
\end{document}
