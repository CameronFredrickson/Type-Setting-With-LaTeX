\documentclass[11pt]{article}

\usepackage{pgfornament,microtype}
\usepackage[pdfborder={0 0 0}, pdftitle={Math 143 Syllabus}]{hyperref}
\usepackage[margin=1in, bottom = .5in]{geometry}
\usepackage[absolute]{textpos}

% These next four lines require comiling with xelatex.  Remove to use pdflatex.
\usepackage{fontspec}
\setmainfont{TeX Gyre Pagella}
\setmonofont{InconsolataGo-Regular.ttf}
\setsansfont{Spring.ttf}

\setlength{\TPHorizModule}{1in}
\setlength{\TPVertModule}{1in}
\setlength{\parindent}{0ex}
\setlength{\parskip}{1.5ex}
\setlength{\topsep}{0ex}

\pagestyle{empty}

\begin{document}

\begin{textblock}{1}(0,0) \pgfornament[height=1in, opacity=.75]{61} \end{textblock}
\begin{textblock}{1}(7.5,0) \pgfornament[height=1in, opacity=.75]{62} \end{textblock}
\begin{textblock}{1}(0,10) \pgfornament[symmetry=h, height=1in, opacity=.75]{61} \end{textblock}
\begin{textblock}{1}(7.5,10) \pgfornament[symmetry=h, height=1in, opacity=.75]{62} \end{textblock}

\begin{center}
  \begin{Huge}\textsf{Calculus III}\end{Huge} \\[1ex]
  mathematics 143 syllabus  \\
  \pgfornament[opacity = .75, height = .1in]{88} \\[3ex]
\end{center}

\textbf{Instructor:} Anthony Mendes.   Call me Tony.  If uncomfortable with first names, call me
Dr.\@ Mendes. 

\textbf{Email:} \texttt{aamendes@calpoly.edu}

\textbf{Office hours:} Office hours are held in Building 25, Room 202.  Office hours are at 
\begin{center}
\begin{tabular}[t]{lll}
  3:10--4:00 & Mondays \\
  8:30--9:20 & Thursdays \\
  3:10--4:00 & Thursdays  \\
  2:10--3:00 & Fridays 
\end{tabular}
\end{center}

\textbf{Website}: Assignments, blue book problems, and old quizzes and exams are
online at
\begin{center}
\url{www.calpoly.edu/~aamendes/143.html} 
\end{center}

\textbf{Text}: Our textbook is 
\textsl{Thomas' Calculus, 12th edition}, Addison Wesley, 2010.  This is the textbook 
for all Calculus I, II, III, and IV sections at Cal Poly.

\textbf{Content:}  The course topics are those found in chapters 10--13 of our textbook.  These are
extremely useful technical tools needed for many branches of science and engineering.  

\textbf{Grading:} 
Letter grades are based on blue book exercises, group quizzes, homework exercises, two
midterms, and a final exam. The proportions of the total grade given to each activity are
\begin{center}
\begin{tabular}{r c}
Blue book exercises & 1/20 \\
Group quizzes & 1/20 \\
Homework exercises & 3/20 \\
Midterm exams & 9/20 \\
Final exam & 6/20 \\
\hline
Total score &  20/20
\end{tabular}
\end{center}
Letter grades are given such that total percent scores of
\begin{center}
\begin{tabular}[b]{rl}
  $93$---$100$ & earn an A, \\
  $90$---$92$ & earn an A$-$,
\end{tabular}
\begin{tabular}{ll} 
$88$---$89$ & earn a B$+$, \\
$83$---$87$ & earn a B, \\
$80$---$82$ & earn a B$-$,  
\end{tabular}
\begin{tabular}{ll} 
$78$---$79$ & earn a C$+$, \\
$73$---$77$ & earn a C, \\
$70$---$72$ & earn a C$-$,  
\end{tabular}
\begin{tabular}{ll}
$68$---$69$ & earn a D$+$, \\
$63$---$67$ & earn a D, \\
$60$---$62$ & earn a D$-$.
\end{tabular}
\end{center}
If you are ever concerned about your grade, please visit me!  It often is not 
as dire as one may think.

\textbf{Homework:} 
Homework is important!  
Calculator and computer use is encouraged.  Working with 
classmates is also encouraged since talking about these ideas with others 
is an outstanding way to learn.  High quality work is expected.  Late assignments are
not accepted.  

\textbf{Exams:} Midterm topics will be announced in class and will 
only cover part of the course.  The final exam will be cumulative but may 
emphasize material not covered on the midterms.

\textbf{Dates}:
\begin{tabular}[t]{ll}
  Group Quizzes: & April 11, 18, 25; May 5, 12, 19; June 6. \\
  Midterm Exams: & Friday, April 28 and Friday, May 26. \\
  Final Exams: & The final for the 12pm section is on Wednesday, June 14 at 10:10. \\
  & The final for the 1pm section is on Friday, June 16 at 1:10. 
\end{tabular}

\end{document}

