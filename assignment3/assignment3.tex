% use OttoBib, click quotes on google scholar, or use mathscinet
% compile twice or thrice for table of contents, .bib, index, or label
\documentclass{article}

\usepackage{amssymb,amsmath,amsthm}
\usepackage[pdftitle = {Assignment 3}, pdfauthor = {Cameron Fredrickson}, pdfsubject = {Typesetting}, colorlinks = true, urlcolor = blue, linkcolor = blue, citecolor = blue]{hyperref}

\usepackage{hologo}

\usepackage{makeidx}
\makeindex

\newtheorem{theorem}{Theorem}

\title{Assignment 3}
\author{Math 351}
\date{}

\begin{document}

\maketitle

\tableofcontents
\section{Instructions}

There are two exercises (which were typeset using the theorem environment).
% spacing between lines needs to be accounted for
\noindent\textbf{Exercise 1.} \emph{Recreate this entire document.}\footnote{How meta.}

\noindent\textbf{Exercise 2.} \emph{Create a new document containing a short description
of three of your favorite books, papers, or other publications. Be sure to include
a bibliog-raphy, created using \hologo{BibTeX}.}

An assignment which completes Exercise 2 in an interesting way or makes amusing
use of mathematical typesetting will earn the coveted \LaTeX er of the week distiction.

\subsection[Due Date]{When to turn it in}

Please upload the \verb~.tex~ and \verb~.bib~ source files and the \verb~.pdf~ output files
to your solutions to Assingment 3 on or before Sunday.

\section{Euler was excellent}

Euler proved many statements, such as
\begin{equation}
    \prod_{m=1}^{\infty} \left(1-q^m\right) = \sum_{n=\infty}^{\infty} \left(-1\right)^n q^{\left(3n^2-n\right)/2}. \label{eq1}
\end{equation}
where q is an indeterminate. Equation \eqref{eq1} \index{pentagonal number theorem}is known as Euler's pentagonal number
theorem. Euler also proved theorem 1 below.

\begin{theorem}[The Basel Problem.]
% formatting of the summation needs to be adjusted
We have $\sum\limits_{n=1}^{\infty} \frac{1}{n^2} = \frac{\pi^2}{6}$.
\end{theorem}

Euler's original proof of Theorem 1 makes unjustified assumptions that infinite products and sums behave like finite products and sums, but interesting nonetheless and worth displaying.

\begin{proof}
Using the power series for sin x, we have
\begin{align}
    \frac{\sin x}{x} & = \frac{1}{x} \left(x-\frac{x^3}{3!}-\frac{x^5}{5!}-\cdots \right) \nonumber \\
    & = 1 - \frac{x^2}{3!} - \frac{x^4}{5!} - \cdots \nonumber \\
    & = \left(1-\frac{x}{\pi} \right) \left(1+\frac{x}{\pi} \right) \left( 1 - \frac{x}{\pi} \right) \left( 1 + \frac{x}{\pi} \right)\cdots
\end{align}
where the reasoning\footnote{This reasoning is actually true, just needs further justification.}
\end{proof}

\end{document}